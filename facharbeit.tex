%//cSpell:disable
\documentclass[a4paper,ngerman, headheight=28pt,12pt]{scrartcl}


% PACKAGES
\usepackage[a4paper,left=3cm,right=4cm,top=2.5cm,bottom=2.5cm]{geometry}

\usepackage{fontspec}
\usepackage{babel}
\usepackage{csquotes}
\usepackage{svg}
\usepackage{relsize}
\usepackage{setspace}

\usepackage{lineno}

\usepackage{tabularray}
\usepackage{float}

% Title Spacing

\RedeclareSectionCommand[
  %runin=false,
  afterindent=false,
  beforeskip=.5\baselineskip,
  afterskip=.25\baselineskip]{section}
\RedeclareSectionCommand[
  %runin=false,
  afterindent=false,
  beforeskip=.25\baselineskip,
  afterskip=.125\baselineskip]{subsection}
\RedeclareSectionCommand[
  %runin=false,
  afterindent=false,
  beforeskip=.175\baselineskip,
  afterskip=.0625\baselineskip]{subsubsection}


% Literaturverzeichnis
\usepackage[
backend=biber,
style=alphabetic,
sorting=ynt,
maxbibnames=99
]{biblatex}

\addbibresource{refs_facharbeit.bib}

\DeclareLabelalphaTemplate{
  \labelelement{
    \field[final]{shorthand}
    \field{label}
    \field[strwidth=2,strside=left,ifnames=1]{labelname}
    \field[strwidth=1,strside=left]{labelname}
  }
  \labelelement{
    \field[strwidth=2,strside=right]{year}
  }
}
%//TODO Change font back to calibri
\setmainfont{Calibri}

\MakeOuterQuote{"}

\newcommand{\LongMinus}{–}

%//cSpell:enable
% Die nächsten vier Felder bitte anpassen:
\newcommand{\Titel}{Dezentralisierte asymmetrische \\ Verschlüsselung über Tor} % Titel für Facharbeit
\newcommand{\SubTitel}{Die Lösung für sicheres Messaging?}

\newcommand{\PageTitel}{Dezentralisierte asymmetrische Verschlüsselung \\ über Tor \LongMinus{} Die Lösung für sicheres Messaging?} % Seitentitel für Facharbeit
\newcommand{\Author}{Hendrik Lind}     % Ich
\newcommand{\Department}{Seminarfach Informatik}
\newcommand{\School}{Windthorst-Gymnasium Meppen}
\newcommand{\Country}{Deutschland}
\newcommand{\Abgabe}{20. November 2023}

\newcommand{\thesisDegree}{Facharbeit}
\newcommand{\faculty}{Seminarfach Informatik}
\newcommand{\thesisPlaceDate}{\today}

\newcommand{\vcite}[1]{\cite[vgl.][]{#1}}
\newcommand{\vebd}{[vgl. ebd.]}

\newcommand{\entryn}{\textit{Entry Node\,}}
\newcommand{\relayn}{\textit{Relay Node\,}}
\newcommand{\relayns}{\textit{Relay Nodes\,}}
\newcommand{\exitn}{\textit{Exit Node\,}}
\newcommand{\nodes}{\textit{Nodes\,}}
\newcommand{\node}{\textit{Node\,}}
\newcommand{\onion}{\textit{Onion\,}}
\newcommand{\circuit}{\textit{Circuit\,}}
\newcommand{\circuits}{\textit{Circuits\,}}
\newcommand{\introp}{\textit{Introduction Point\,}}
\newcommand{\introps}{\textit{Introduction Points\,}}
\newcommand{\renp}{\textit{Rendezvous Point\,}}
%//cSpell:disable


% Kopf- und Fußzeilen
\usepackage{scrlayer-scrpage, lastpage}
\setkomafont{pageheadfoot}{\large\textrm}
\lohead{\PageTitel}
\rohead{\Author}
\cfoot*{\thepage{}/\pageref{LastPage}}

% Position des Titels
\usepackage{titling}
\setlength{\droptitle}{-1.0cm}


% Für mathematische Befehle und Symbole
\usepackage{amsmath}
\usepackage{amssymb}
\usepackage{wrapfig}

% Für Bilder
\usepackage{graphicx}
\usepackage{graphbox}

% Für Algorithmen
\usepackage{algpseudocode}

% Für Quelltext
\usepackage{listings}
\usepackage{color}


\graphicspath{ {./img/} }


% 1.5 Line spacing
\setstretch{1.5}

\include{RustLanguage}

% Umlaute erlauben
\lstset{literate=%
  {Ö}{{\"O}}1
  {Ä}{{\"A}}1
  {Ü}{{\"U}}1
  {ß}{{\ss}}1
  {ü}{{\"u}}1
  {ä}{{\"a}}1
  {ö}{{\"o}}1
}
%end


\definecolor{mygreen}{rgb}{0,0.6,0}
\definecolor{mygray}{rgb}{0.5,0.5,0.5}
\definecolor{mymauve}{rgb}{0.58,0,0.82}
\lstset{
  keywordstyle=\color{blue},commentstyle=\color{mygreen},
  stringstyle=\color{mymauve},rulecolor=\color{black},
  basicstyle=\footnotesize\ttfamily,numberstyle=\tiny\color{mygray},
  captionpos=b, % sets the caption-position to bottom
  keepspaces=true, % keeps spaces in text
  numbers=left, numbersep=5pt, showspaces=false,showstringspaces=true,
  showtabs=false, stepnumber=2, tabsize=2, title=\lstname{}
}
\lstset{language=Rust}          % Set your language (you can change the language for each code-block optionally)

% Diese beiden Pakete müssen zuletzt geladen werden
\usepackage[hidelinks]{hyperref} % Anklickbare Links im Dokument
\usepackage{cleveref}


% Titlepage required things


% Necessary packages for the titlepage:
\usepackage{tikz}
\usetikzlibrary{calc}
%\usepackage{graphicx}
% \usepackage{newtxtext}
%\usepackage{float}
\usepackage{comment}
% This command changes the font style where SLU promotes Arial
%\newenvironment{myfont}{\fontfamily{phv}\selectfont}{\par}




% Facharbeit

\begin{document}
\include{titlePage}
\tableofcontents
\setcounter{page}{0}
\thispagestyle{empty}
\vspace{0.5cm}
\pagebreak


%//cSpell:enable
\linenumbers{}
\modulolinenumbers[5]
\section{Einleitung}
%//SECTION Einleitung
Russland, China, Iran. In all diesen totalitären Staaten herrscht eine starke Zensur \vcite{AmnReport}. Rund 1,7 Milliarden Menschen sind allein nur in diesen drei Staaten von der Einschränkung der Meinungsfreiheit betroffen \vcite{UnPop}. Wie können Bürger
dieser Staaten ihre Meinung also verbreiten und andere Staaten auf
%//TODO hier staatskritisch ist falsches wort mir fällt aber das richtige nicht ein
staatskritische Probleme aufmerksam machen ohne sich selber in Gefahr zu bringen?
%//!SECTION
%//SECTION Problemstellung
\\
%//TODO Hier nochmal nach einer anderen Quelle suchen, die passt nicht 100%ig
Bei herkömmlichen Messengern, wie WhatsApp, Signal und co., braucht die Außenwelt die Telefonnummern der im totalitären Staat wohnenden Bürgern und Reportern, um diese zu kontaktieren. Allerdings könnte ein totalitärer Staat, sich als Empfänger ausgeben, sodass Bürger/Reporter ihre private Nummer an den Staat überreichen und dieser somit jene Nummer rückverfolgen kann \vcite{LocPolice}.
Und genau hier liegt das Problem: Bürger und Reporter können nicht durch alltägliche Messenger mit der Außenwelt kommunizieren, da der Staat deren Nummer zurückverfolgen kann und somit weiter die Meinungsfreiheit einschränkt und unterbindet \vebd.
\\
Durch die zentrale Infrastruktur, welche die meisten Messenger, wie zum Beispiel WhatsApp und Signal verwenden, ist es für totalitäre Staaten, wie China, möglich, die IP-Adressen jener Server zu blockieren und somit für Bürger und Reporter unzugänglich zu machen \vcite{ChinaFirewall,CentralizedWhatsapp}.
\\
%//!SECTION
%//SECTION Lösungsvorschlag / Ziel
Ein dezentralisierter Messenger, welcher Ende-zu-Ende verschlüsselt ist und über das Tor-Netzwerk kommuniziert, könnte bei diesen Problemen eine Lösung sein. Die Frage, ob ein solcher Messenger die Lösung für Bürger eines totalitären Staates ist, soll in dieser Arbeit geklärt werden.
%//!SECTION
\\
%//SECTION E2EE Überleitung zu Kapitel-Auflistung
Um diese Frage beantworten zu können, beschäftigt sich diese Arbeit in dem zweiten Kapitel mit der asymmetrischen Verschlüsselung, welche benötigt wird um die Ende-zu-Ende-Verschlüsselung (E2EE) umzusetzen und die Definition der E2EE, sowie der Sicherheitsbetrachtung der asymmetrischen Verschlüsselung \vcite{E2EE-Method}. Die Arbeit geht dabei nicht auf weitere Padding-Verfahren ein.
Eine mögliche Lösung, um eine Anonymität über das Internet zu gewährleisten wird in Kapitel drei vorgeschlagen, wobei das Tor-Netzwerk eine wichtige Rolle spielt.
Diese Arbeit befasst sich im vierten Kapitel mit einer Dezentralisierung der Infrastruktur, um eine weitere Sicherheitsebene zu schaffen.
Zuletzt werden im fünften Kapitel die Vor- und Nachteile eines solchen Messengers betrachtet, im sechsten Kapitel wird eine mögliche Umsetzung des Messengers beschrieben und im siebten Kapitel wird ein Fazit gezogen.
%//!SECTION


\section{Asymmetrische Verschlüsselung}
%//SECTION E2EE und asymmetrische Verschlüsselung
Um einen sicheren Nachrichtenaustausch zu gewährleisten, wird in dieser Arbeit die E2EE implementiert. Bei der E2EE wird von dem Sender die Nachricht, bevor sie an den Empfänger geschickt wird, verschlüsselt \vcite{E2EE}. Dazwischenliegende Akteure, wie zum Beispiel Server oder mögliche Angreifer, können demzufolge die Nachricht nicht lesen \vebd. \textbf{Nur} der Empfänger der Nachricht kann diese auch entschlüsseln. Als Ent- und Verschlüsselungsverfahren der Nachrichten wird die asymmetrische Verschlüsselung verwendet \vebd. Diese Arbeit beschränkt sich bei der asymmetrischen Verschlüsselung auf das RSA-Verfahren.
%//!SECTION
\subsection{Grundlagen}
%//SECTION Grundlagen
Grundsätzlich gibt es bei der asymmetrischen Verschlüsselung ein Schlüsselpaar (Keypair), welches aus einem privaten Schlüssel (private key) und einem öffentlichen Schlüssel (public key) besteht \vcite{Rsa-Basics}. Diese beiden Schlüssel hängen mathematisch zusammen, sodass der öffentliche Schlüssel Nachrichten \textbf{nur} verschlüsseln aber nicht entschlüsseln kann \vebd. \textbf{Nur} der zum Schlüsselpaar dazugehörige private Schlüssel ist in der Lage, die verschlüsselte Nachricht wieder zu entschlüsseln (siehe \cref{fig:E2EE}) \vebd.

\begin{figure}[h]
  \centering
  \includegraphics[width=0.75\textwidth]{Briefkasten-asymm.png}
  \caption{Jeder Sender kann mit dem öffentlichen Schlüssel die Nachricht "verschlüsseln" (also eine Nachricht in den Briefkasten werfen), aber nur der Empfänger kann den Briefkasten mit seinem privaten Schlüssel öffnen und somit die Nachricht herausnehmen\vcite{fig:Rsa-Cryptography} \label{fig:E2EE}}
\end{figure}
%//!SECTION

%//SECTION - Mathematische Betrachtung zu RSA
\subsection{Mathematische Betrachtung}
Alle Variablen der folgenden Berechnungen liegen im Bereich $\mathbb{N}$ \vcite{RsaGenCond}. \\
Für die Generierung des Schlüsselpaares benötigen wir zuerst zwei große zufällige Primzahlen, $P$ und $Q$ \vebd. Daraus ergibt sich $n = P * Q$, wobei $P \neq Q$, sodass $P$ bzw. $Q$ nicht durch $\sqrt{n}$ ermittelt werden kann \vebd. Der private Schlüssel besteht aus den Komponenten $\{ n, d \}$ währenddessen der öffentliche Schlüssel aus $\{ n, e \}$ besteht \vcite{RsaVariables}.
%//!SECTION
%//SECTION - Eulersche Phi-Funktion
\subsubsection{Eulersche Phi-Funktion}
Die Eulersche Phi-Funktion spielt eine wichtige Rolle in dem RSA-Verfahren \vcite{TotientFuncMultiplicative}. Grundsätzlich gibt $\phi(x)$ an, wie viele positive teilerfremde Zahlen bis $x$ existieren (bei wie vielen Zahlen der größte gemeinsamer Teiler ($\gcd$) $1$ ist) \vcite{EulersTotientFunction}. Somit ergibt $\phi(6) = 2$  oder bei einer Primzahl $\phi(7) = 7 - 1 = 6$ somit $\phi(x) = x-1$, wenn $x$ eine Primzahl ist, da jede Zahl kleiner als $x$ teilerfremd sein muss \vcite{TotientFuncMultiplicative}.
\begin{equation*}
  \begin{aligned}
    \phi(n) & = \phi(P \cdot Q)                                                \\
    \phi(n) & = \phi(P) \cdot \phi(Q)                                          \\
    \phi(P) & = P -1                                          & \phi(Q) = Q -1 \\
    \phi(n) & = \left(P - 1 \right) \cdot \left( Q - 1\right)
  \end{aligned}
\end{equation*}
%//!SECTION
\subsubsection{Generierung des Schlüsselpaares}
Sowohl der private als auch der öffentliche Schlüssel besteht unter anderem aus folgender Komponente: $n = P \cdot Q$ \vcite{RsaMaths1}.
Für den öffentlichen Schlüssel benötigen wir die Komponente $e$, die zur Verschlüsselung einer Nachricht verwendet wird \vebd. $e$ ist hierbei eine zufällige Zahl, bei welcher folgende Bedingungen gelten \vebd:
\begin{equation*}
  e = \begin{cases}
    1 < e < \phi(n)      \\
    \gcd(e, \phi(n)) = 1 \\
    \text{$e$ kein Teiler von $\phi(n)$}
  \end{cases}
\end{equation*}
Mit der errechneten Komponente $e$, welche Nachrichten verschlüsselt, kann der öffentliche Schlüssel nun an den Sender übermittelt werden.

Um den privaten Schlüssel zu berechnen benötigen wir die Komponente $d$, welche zur Entschlüsselung verwendet wird \vcite{RsaEncryptionDecryption}.
\begin{equation*}
  \begin{aligned}
    \phi(n)   & = (P-1)(Q-1)     \\
    e \cdot d & = 1 \mod \phi(n)
  \end{aligned}
\end{equation*}

\subsection{Sicherheit}
Um die Sicherheit des RSA-Verfahrens betrachten zu können, müssen wir nun den Ver/-Entschlüsselungsvorgang betrachten.
\begin{equation*}
  \begin{aligned}
    c & = m^e \mod n & \text{Verschlüsselung zu $c$ mit $m$ als Nachricht}    \\
    m & = c^d \mod n & \text{Umkehroperation (Entschlüsslung) von $c$ zu $m$}
  \end{aligned}
\end{equation*}
Um die verschlüsselte Nachricht $c$ zu entschlüsseln, bräuchte ein Angreifer die Komponente des privaten Schlüssels $d$. $d$ ist allerdings mit einem starken Rechenaufwand verbunden, da, wie schon vorher bereits gezeigt, dafür $\phi(n)$ kalkuliert werden müsste. Somit also eine Primfaktorzerlegung von $n$ benötigt wird \vcite{EulersTotientFunction}. Bei der Verschlüsselung von $m$ zu $c$ liegt eine Trapdoor-Einwegfunktion vor \vcite{RsaTrapdoor}. Das bedeutet, dass es zwar leicht ist $f(x) = i$ zu berechnen (bei RSA: Verschlüsselung), es jedoch unmöglich ist von $i$ auf den Ursprungswert $x$ zu schließen, ohne dass weitere dafür notwendigen Komponente bekannt sind (bei RSA wäre die benötigte Komponente $d$) \vebd.
\begin{figure}[h]
  \centering
  \includesvg[width=0.5\textwidth]{img/Trapdoor_permutation.svg}
  \caption{Die Trapdoor-Einwegfunktion bildlich dargestellt\vcite{fig:TrapdoorPermutation} \label{fig:TrapdoorFunc}}
\end{figure}

Wichtig bei dem RSA-Verfahren ist, dass die Länge von $n$ (die Schlüssellänge) mindestens 3000 Bit betragen sollte, da sonst die Primfaktorzerlegung von $n$ mit modernen Computern möglich sein könnte \vcite{RsaKeyLength}.

%/REVIEW - Ist das wirklich nötig? Oder kann ich mir das sparen? (ja schon)
\subsection{Vergleich zur symmetrischen Verschlüsselung}
Bei der symmetrischen Verschlüsselung wird der gleiche Schlüssel sowohl für die Verschlüsselung als auch für die Entschlüsselung verwendet \vcite{GeneralSymmetricCryptography}.
%/REVIEW - Wirklich padding mit einbeziehen? (meinte er brauch ich nich)
Im Vergleich zu der asymmetrischen Verschlüsselung, ist die symmetrische Verschlüsselung schneller und keine Beschränkung des Chiffretextes \vcite{RsaAESAnalysis, OpensslRsaMaxLength}. Jedoch muss der Schlüssel der symmetrischen Verschlüsselung sicher an den jeweils anderen Kommunikationspartner übermittelt werden, um Nachrichten zu entschlüsseln \vebd.

\section{Anonymität des Tor-Netzwerkes}
Der zweite zentrale Aspekt dieser Arbeit ist die Anonymität über das Internet.
%//REVIEW für den Satz danach Quelle?
Bei normalen Routing, wie wir es tagtäglich nutzen, kommuniziert der Client direkt mit dem Zielserver. Der Zielserver kann hierbei die IP-Adresse des Clients sehen, somit ist eine Rückverfolgung möglich \vcite{LocPolice,TCP_IP}. Und genau bei diesem Problem setzt das Tor-Netzwerk an. Das Tor-Netzwerk besteht hierbei aus vielen \nodes, welche eingehende Tor-Verbindungen akzeptieren und weiterverarbeiten. Damit ein Client überhaupt eine Anfrage über das Netzwerk verschicken kann, sucht er sich zunächst einen Pfad durch das Netzwerk, genannt \circuit \vcite{TorCircuits}. Dieser \circuit besteht dabei meist aus drei \nodes und ist für 10 Minuten gültig, bis der Client das Circuit erneuert (also einen neuen Pfad "sucht") \vcite{FAQCircuitLifetime}. Der \entryn, einer \relayn und einer \exitn, worüber später Anfragen an die Zielserver geschickt werden können \vebd.
Dabei verschlüsselt der Client verschlüsselt die eigentliche Anfrage mehrmals, hüllt sie also in ein mehrere "Schalen" ein, welche eine \onion bilden, und leitet diese über den \circuit an den Zielserver weiter\vcite{TorDesign}. Anfangs wird die \onion von dem Tor-Client an die \entryn geschickt \vebd. Bei jeder \node, also auch der \entryn, wird eine Schale der \onion "geschält" (die Nachricht also einmal entschlüsselt), welche Informationen zu dem nächsten Knotenpunkt enthält, somit schickt jene Node, die Anfrage an den nächsten Knotenpunkt weiter\vcite{TorStructure2}.
%//TODO Checken ob das wirklich in der quelle so steht, sollte eig
Sobald die Anfrage bei der \exitn angekommen ist, entfernt diese die letzte "Schale" der Anfrage, welche nun vollständig entschlüsselt ist (wenn die eigentliche Anfrage nicht mit HTTPS verschlüsselt wurde), und schickt diese an den Zielserver \vebd. Nur die \entryn weiß somit die reale IP-Adresse des Clients und \textbf{nur} die letzte Node (\exitn) weiß, an welchen Zielserver die Anfrage geschickt wurde \vebd. Wichtig hierbei ist allerdings, dass die \entryn nicht weiß, was der Zielserver der Anfrage ist \vebd. Die \exitn weiß im Gegenzug die reale IP-Adresse des Clients nicht. Allerdings könnte die \exitn die IP-Adressen der Zielserver sehen oder möglicherweise Informationen der Nutzer auslesen (wenn die Anfrage nicht mit HTTPS verschlüsselt wurde) \vcite{TorExitNodeVulnerability}. Hier herrscht also eine Schwachstelle des Tor-Netzwerkes.
% Die IP-Adresse des Zielservers ist jedoch nicht im Tor-Netzwerk "versteckt" und kann somit leicht durch den Staat rückverfolgt werden.

\begin{figure}[h]
  \centering
  \includegraphics[width=0.65\textwidth]{tor_vertical.png}
  \caption{Tor Entschlüsselungslayer \vcite{fig:Tor-Structure} \label{fig:TorStructure}}
\end{figure}
\subsection{Onion Services}
Onion Services lösen dieses Problem, da diese nicht auf die \exitn angewiesen sind und nur innerhalb des Tor-Netzwerkes mit anderen Tor-Clients kommunizieren \vcite{TorOnionServiceTalk}. Sie nicht über das normale Internet erreichbar, wie die Zielserver im vorherigen Beispiel, sondern nur über das Tor-Netzwerk \vcite{TorOnionService}. Bei dem Onion Service müssen im Kontrast zu normalen öffentlichen Servern keine Ports geöffnet werden, damit ein Client sich mit dem Server verbinden kann, da der Onion Service direkt mit dem Tor-Netzwerk über ausgehende Verbindungen kommuniziert (auch bekannt als \textit{NAT punching}) und darüber sämtliche Daten geleitet werden \vebd. Für Außenstehende sieht der Onion Service also aus, wie als wäre er ein normaler Tor-Client \vebd.

\subsubsection{Verbindungsaufbau}
Zunächst generiert der Onion Service ein Schlüsselpaar, bestehend aus einem öffentlichen und einem privaten Schlüssel \vcite{GeeksOnionService}. Unter anderem wird nun aus dem öffentlichen Schlüssel die Adresse des Onion Services generiert und endet mit ".onion" \vebd. Ein Beispiel für eine solche Adresse ist die der Suchmaschine DuckDuckGo: \\
\textit{duckduckgogg42xjoc72x3sjasowoarfbgcmvfimaftt6twagswzczad.onion} \vcite{DuckDuckGoLink}.
%/REVIEW sehr oft gleiche Quelle (Geertsema meinte is okay)
Der Onion Service sich nun mit dem Tor-Netzwerk wie ein normaler Client über einen \circuit, welcher aus drei \nodes besteht \vcite{TorOnionService}. Der Service sendet eine Anfrage an das letzte Relay im \circuit, sodass es als \introp dient und etabliert eine Langzeitverbindung jenem (der \circuit erneuert sich also nicht alle 10 Minuten wie bei dem Tor-Client) \vebd. Dieser Vorgang wiederholt sich zwei Mal, bis der Onion Service drei \introps auf drei verschiedenen Circuits gefunden hat \vebd. Damit andere Clients den Onion Service erreichen können, erstellt der Onion Service einen \textit{Onion Service descriptor}, welcher die Adressen der \introps und Authentifizierungsschlüssel enthält, signiert diesen mit seinem privaten Schlüssel und schickt den \textit{descriptor} an die Directory Authority \vcite{TorSpecDerivingKeys, TorSpecDirectoryInf}. Die Directory Authority ist ein Server, welcher Informationen über das Tor-Netzwerk, wie zum Beispiel die des Onion Services, speichert und verteilt \vcite{TorDirectoryAuthority}. Im Sinne dieser Arbeit, hat der Onion Service sich nun von Person A mit dem Tor-Netzwerk verbunden, jedoch braucht es noch Person B, welche sich über ihren Tor-Client mit dem Onion Service der Person B verbindet. Damit der Tor-Client von Person A sich allerdings verbinden kann, fragt der Client nun die Directory Authority an den signierten \textit{descriptor} des Onion Services von Person B an den Client zu schicken \vcite{TorStructure}. Der Client besitzt nun die Adressen der \introps und die Signatur des \textit{descriptors}, sodass dieser mit dem öffentlichen Schlüssels der Onion Adresse die Signatur überprüfen kann \vebd. Der Client generiert nun 20 zufällige Bytes (Secret) und schickt diese an eine zufällig ausgewählte \relayn, welche nun als \renp dient \vcite{TorSpecRendezvous}. Das Secret wird von dem Client auch an eine von den \introps geschickt, sodass der Onion Service mit dem gleichen Secret eine Verbindung zu dem \renp aufbauen kann \vcite{TorSpecIntroP}. Der \renp leitet, wenn der Client und der Onion Service über deren \circuits miteinander verbunden sind, die Nachrichten zwischen den beiden weiter \vcite{TorSpecRendezvous}. Der Client und der Onion Service kommunizieren nun also nur über das Tor-Netzwerk miteinander und sind nicht mehr auf die \exitn angewiesen.

\subsection{Vor- und Nachteile}
%//TODO Nochmal Studio durchlesen, ist noch nicht schlüssig
Vorteile des Tor-Netzwerkes (bezogen auf Onion Services) sind zum Beispiel, wie bereits genannt, die Anonymität. Das Routing über das Tor-Netzwerk ist im Vergleich zum normalen Routing um das 120-fache langsamer \vcite{TorPerformance}. Ein möglicher Angriff gegen die Anonymität des Onion Services ist die des "SignalCookies", welcher auf einen Fehler im Protokoll des Tor-Netzwerkes zurückzuführen ist \vcite{OnionServiceDiscovery}. Hierbei wird ein zufälliger Cookie, bestehend aus 20 Bytes generiert und zwischengespeichert \vebd. Der Client sendet nun eine Anfrage in das Tor-Netzwerk, dass der Client mit dem Onion Service eine Verbindung aufbauen will und hängt den Cookie an die Anfrage an. Der Onion Service baut nun, wie bereits beschrieben eine Verbindung über einen \circuit auf und einigt sich auf einen \renp mit dem Client, jedoch schickt er während der Anfrage an den \renp aufgrund eines Protokollfehlers des Tor-Netzwerk den vorher gegebenen Cookie des Clients mit \vebd. Wenn der \renp nun von einem Angreifer kontrolliert wird, kann dieser den Cookie des Clients mit dem Cookie des Onion Services vergleichen und somit den Onion Service identifizieren, wodurch sich von hieraus vorgearbeitet werden kann, sodass die \entryn des Onion Services identifiziert wird und kompromittiert werden kann \vebd.

\section{Dezentralisierung}
\subsection{Sicherheit}


\section{Vor- und Nachteile}
\subsection{Kriminalität}
\subsection{freie Meinungsäußerung}

\section{Umsetzung des Messengers}

\section{Fazit}

\pagebreak
\nolinenumbers{}
\printbibliography[notkeyword={figure}]

\pagebreak
\printbibliography[heading=subbibliography,title={Anhang},keyword={figure}]
Quellcode: \href{https://github.com/sshcrack/enkrypton}{https://github.com/sshcrack/enkrypton}

\end{document}
